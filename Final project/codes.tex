\section{Source Code}
The following source code in R and C++ extensions make up our package $parallelBS$. It can be loaded using the following R command:
\begin{verbatim}
library(parallelBS)
\end{verbatim}

\subsection{Driver}
This is how the function in our package is run. Prior to running our function, the user must also link the parallel and Rcpp libraries.
\lstset{language=R}
\begin{lstlisting}
#bs function call
bs = function(x, degree=3, interior.knots=NULL,
              intercept=FALSE, Boundary.knots = c(0,1),
              type="serial", ncls=2)
{
    #input check
    if(missing(x))
    stop("You must provide x")
    if(degree < 1)
    stop("The spline degree must be at least 1")

    #input preprocessing
    Boundary.knots = sort(Boundary.knots)
    interior.knots.sorted = NULL
    if(!is.null(interior.knots))
        interior.knots.sorted = sort(interior.knots)

    #formation of knots
    knots = c(rep(Boundary.knots[1], (degree+1)),
                   interior.knots.sorted,
                   rep(Boundary.knots[2], (degree+1)))
    K = length(interior.knots) + degree + 1
    lenx = length(x)

    #parallel calls
    if(type == "openmp") {
        print("run type: openmp")
        dyn.load("bs_omp.so")
        Bmat = .Call("formMatrix",x,degree,knots,K,lenx)
    }
    else if (type=="cuda") {
        print("run type: cuda")
        dyn.load("bs_cuda.so")
        Bmat = .Call("formMatrix",x,degree,knots,K,lenx)
    }
    else if(type=="snow") {
        print("run type: snow")
        cls = makePSOCKcluster(rep("localhost", ncls))
        print(cls)
        Bmat = formMatrixSnow(cls, x, degree, knots, K)
    }
    else if(type=="serial") {
        print("run type: serial")
        Bmat = matrix(0,length(x),K)
        for(j in 1:K)
            Bmat[,j] = basis(x, degree, j, knots)
    }
    else {
        print("Incorrect run type - serial used instead")
        #do serial
        Bmat = matrix(0,length(x),K)
        for(j in 1:K)
            Bmat[,j] = basis(x, degree, j, knots)
    }
    #output postprocessing
    if(any(x == Boundary.knots[2]))
        Bmat[x == Boundary.knots[2], K] = 1
    if(intercept == FALSE)
        return(Bmat[,-1])
    else
        return(Bmat)
} #end bs function

\end{lstlisting}

\subsection{Serial R}
\lstset{language=R}
\begin{lstlisting}
#Serial R code
#exploits Cox-de Boor recursion formula
Basis = function(x, degree, i, knots){
    #recursion base case
    if(degree == 0)
        B = ifelse((x>=knots[i])&(x<knots[i+1]),1,0)
    else{
        #alpha1 computation
        if((knots[degree+i]-knots[i]) == 0)
            alpha1 = 0
        else
            alpha1 = (x-knots[i])/
                     (knots[degree+i]-knots[i])
        #alpha2 computation
        if((knots[i+degree+1] - knots[i+1]) == 0)
            alpha2 = 0
        else
            alpha2 = (knots[i+degree+1]-x)/
                     (knots[i+degree+1]-knots[i+1])
        B = alpha1*basis(x,(degree-1),i,knots)+
            alpha2*basis(x,(degree-1),(i+1),knots)
    } #end else
    return(B)
} #end function
\end{lstlisting}

\subsection{OpenMP} %completed
\lstset{language=C++}
\begin{lstlisting}
//C++ interface employing OpenMP
//project_omp.cpp

#include <Rcpp.h>
#include <omp.h>
using namespace Rcpp;

NumericVector basis(NumericVector x, int degree, int i,
                    NumericVector knots, int lenx)
{
    //memory allocation
    NumericVector B(lenx), alpha1(lenx), alpha2(lenx);
    if(degree==0) //recursion base case
        B = wrap(ifelse((x >= knots[i])
                 &(x < knots[i+1]), 1, 0));
    else{
        //alpha1 computation
        if((knots[degree+i] - knots[i]) == 0)
            alpha1 = rep(0,lenx);
        else
            alpha1 = wrap((x-knots[i])/
                           (knots[degree+i]-knots[i]));
        //alpha2 computation
        if((knots[i+degree+1] - knots[i+1]) == 0)
            alpha2 = rep(0,lenx);
        else
            alpha2 = wrap((knots[i+degree+1]-x)/
                          (knots[i+degree+1]-knots[i+1]));
        B = wrap(alpha1*basis(x,(degree-1),i,knots,lenx)+
                 alpha2*basis(x,(degree-1),(i+1),knots,lenx));
    }//end else
    return B;
} //end basis function

RcppExport SEXP formMatrix(SEXP x_, SEXP degree_, SEXP knots_,
                           SEXP k_, SEXP lenx_)
{
    //convert data types from R to C++
    int degree = as<int>(degree_),
        k = as<int>(k_), lenx = as<int>(lenx_);
    //SEXP to NumericVector:
    //http://dirk.eddelbuettel.com/code/rcpp/Rcpp-quickref.pdf
    NumericVector x(x_);
    NumericVector knots(knots_);

    //output variable allocation:
    //matrix: lenx rows, k columns filled with 0
    NumericMatrix out(lenx,k);

    #pragma omp parallel for
    for(int j = 0; j < k; j++) {
        //R equivalent: for(j in 1:k)
        //Reference jth column; changes propagate matrix
        NumericMatrix::Column jvector = out(_,j);
        #pragma omp critical
        jvector = basis(x, degree, j, knots, lenx);
    }//end for(j)
    return out;
} //end matrix function
\end{lstlisting}

\subsection{CUDA}
\lstset{language=C++}
\begin{lstlisting}
#include <Rcpp.h>
#include <cuda.h>

using namespace Rcpp;

//CUDA kernel function: base case
__global__ void degree0(float *B, float *knots,
                        float *x, int i)
{//current thread
    int me = blockIdx.x * blockDim.x + threadIdx.x;
    B[me] = ((x[me] >= knots[i]) &&
             (x[me] < knots[i+1])) ? 1 : 0;
}//end kernel

NumericVector basis(NumericVector x, int degree, int i,
                    NumericVector knots, int lenx)
{
    NumericVector B(lenx); // output variable
    NumericVector alpha1(lenx), alpha2(lenx);

    if(degree==0){
        //blocks and threads for GPU
    int nthreads = min(lenx, 500),
        nblocks = ceil(lenx/nthreads);
    //set up grid/block dimensions
    dim3 dimGrid(nblocks,1),
         dimBlock(nthreads,1,1);
    float *dB, *dknots, *dx;
    //copy to the GPU
    cudaMemcpy(dB,B,lenx,cudaMemcpyHostToDevice);
    cudaMemcpy(dknots, knots, knots.size(), cudaMemcpyHostToDevice);
    cudaMemcpy(dx, x, lenx, cudaMemcpyHostToDevice);
    degree0<<<dimGrid,dimBlock>>>(dB,dknots,dx,i);
    //copy kernel results back to B
    cudaMemcpy(B,dB,lenx,cudaMemcpyDeviceToHost);
    B = wrap(ifelse((x >= knots[i]) & (x < knots[i+1]), 1, 0));
    } //end if
    else{
        //alpha1 computation
        if((knots[degree+i] - knots[i]) == 0)
            alpha1 = rep(0,lenx);
        else
            alpha1 = wrap((x - knots[i])/(knots[degree+i] - knots[i]));
        //alpha2 computation
        if((knots[i+degree+1] - knots[i+1]) == 0)
            alpha2 = rep(0,lenx);
        else
            alpha2 = wrap((knots[i+degree+1] - x)/
                          (knots[i+degree+1] - knots[i+1]));
         B = (alpha1 * basis(x, (degree - 1), i, knots, lenx))
             +(alpha2 * basis(x, (degree - 1), (i + 1), knots, lenx));
    }//end else
    return B;
} //end basis function

RcppExport SEXP formMatrix(SEXP x_, SEXP degree_, SEXP knots_,
                           SEXP k_, SEXP lenx_)
{
    //convert data types from R to C++
    int degree = as<int>(degree_),
    k = as<int>(k_), lenx = as<int>(lenx_);
    //SEXP to NumericVector
    NumericVector x(x_);
    NumericVector knots(knots_);
    //output variable allocation:
    NumericMatrix out(lenx,k);
    //blocks and threads for GPU
    int nthreads = min(lenx, 500);
    int nblocks = ceil(lenx/nthreads);
    //set up grid/block dimensions
    dim3 dimGrid(nblocks,1), dimBlock(nthreads,1,1);
    float *dB, *dknots, *dx;
    //make space on GPU
    cudaMalloc((void **) &dB,lenx*sizeof(float));
    cudaMalloc((void **) &dknots, knots.size()*sizeof(float));
    cudaMalloc((void **) &dx, lenx*sizeof(float));
    for(int j = 0; j < k; j++) { //R equivalent:  for(j in 1:k){
        //Reference the jth column; changes propagate to matrix
        NumericMatrix::Column jvector = out(_,j);
        jvector = basis(x, degree, j, knots, lenx);
    }//end for(j)
    cudaFree(dB);
    cudaFree(dknots);
    cudaFree(dx);
    return out;
} //end matrix function



\end{lstlisting}

\subsection{SNOW} %completed
\lstset{language=R}
\begin{lstlisting}
#SNOW R code
#applies a cluster to each column of the final matrix
formMatrix = function(cls, x, degree, knots, K){
    sequence = 1:K
    #cluster function - returns column of the matrix
    basis = function(i,degree){
        #recursion base case
        if(degree == 0){
            B = ifelse((x>=knots[i])&(x<knots[i+1]),1,0)
        }
        else{
            #alpha1 computation
            if((knots[degree+i]-knots[i]) == 0)
                alpha1 = 0
            else
                alpha1 = (x-knots[i])/
                         (knots[degree+i]-knots[i])
            #alpha2 computation
            if((knots[i+degree+1] - knots[i+1]) == 0)
                alpha2 = 0
            else
                alpha2 = (knots[i+degree+1]-x)/
                         (knots[i+degree+1]-knots[i+1])
            B = alpha1*basis(i,(degree-1))+
                alpha2*basis((i+1),(degree-1))
        } #end else
        return(B)
    } #end cluster function
    #have each cluster obtain a column of the matrix
    jvector = clusterApply(cls,sequence,basis,degree)
    #form the final matrix to return
    Bmat = Reduce(cbind,jvector)
}
\end{lstlisting}
