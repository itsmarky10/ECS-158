\documentclass{article}
\usepackage[utf8]{inputenc}
\usepackage{amsfonts}
\usepackage{amsmath}
\usepackage{amsthm}
\usepackage{graphicx}
\usepackage{pdfpages}
\usepackage{algorithm}

\usepackage{algpseudocode}
\title{ECS158 Homework \#2 \\ README}
\author{Brian Vargas, Max Matsumoto, Michael Banzon, Mark Fan}
\date{February 13, 2015}

\setlength\parindent{0pt}

\begin{document}
\maketitle

In running our \emph{rmandel} function, we compared the time elapsed for using different scheduling in the OpenMP sections of the C++ code in attempt to find which is best to use in practice. Alongside varying chunk sizes, we explored timings for the four different scheduling types: static, dynamic, auto, and guided.
\\ \\
The following tables represent our results in the timings. The following parameters were used: -4, 4, -4, 4, .01, 1000
\\ \\
\textbf{Static} \\
\begin{tabular}{l|l l l l l}
\hline Chunk Size&4&8&16&256&65536 \\ \hline
Trial 1&0.591&0.591&0.591&0.482&0.482 \\
Trial 2&0.594&0.625&0.481&0.594&0.590 \\
Trial 3&0.481&0.483&0.600&0.482&0.483 \\
Trial 4&0.485&0.482&0.481&0.596&0.481 \\
Average&0.573&0.545&0.538&0.538&0.509 \end{tabular}

\textbf{Dynamic} \\
\begin{tabular}{l|l l l l l}
\hline Chunk Size&4&8&16&256&65536 \\ \hline
Trial 1&0.482&0.590&0.595&0.481&0.481 \\
Trial 2&0.594&0.592&0.482&0.590&0.485 \\
Trial 3&0.594&0.481&0.594&0.590&0.602 \\
Trial 4&0.481&0.595&0.608&0.592&0.595 \\
Average&0.537&0.564&0.596&0.564&0.540 \end{tabular}

\textbf{Auto} \\
\begin{tabular}{l|l l l l l}
\hline Chunk Size&4&8&16&256&65536 \\ \hline
Trial 1&0.482&0.603&0.593&0.594&0.591 \\
Trial 2&0.482&0.482&0.482&0.483&0.482 \\
Trial 3&0.591&0.592&0.591&0.593&0.594 \\
Trial 4&0.482&0.594&0.482&0.484&0.595 \\
Average&0.509&0.567&0.537&0.538&0.565
\end{tabular}

\newpage
\textbf{Guided} \\
\begin{tabular}{l|l l l l l}
\hline Chunk Size&4&8&16&256&65536 \\ \hline
Trial 1&0.637&0.481&0.590&0.595&0.590 \\
Trial 2&0.591&0.594&0.591&0.597&0.486 \\
Trial 3&0.594&0.605&0.482&0.594&0.591 \\
Trial 4&0.543&0.594&0.481&0.590&0.481 \\
Average&0.603&0.568&0.563&0.594&0.537
\end{tabular}
\\ \\
These tables can be summarized by looking at the following plot of average timings. Note the following color coding: static-red, dynamic-blue, auto-black, guided-green.
\begin{center} \includegraphics[scale=.48]{plot1.jpeg} \end{center}

By observing the average times for the chunk size chosen, with the various types of
scheduling, it appears that our best time was achieved when our chunk size was
$2^{16}=65536$ and under the static schedule. This was used to create the optimal function mandelopt().
\\ \\
An interesting observation is that dynamic had the greatest decrease in
time from chunk size 4 to chunk size 8. Perhaps with much larger chunks, guided
ends up the same as or better than static.
\\ \\
Still, we should conclude that our optimal set up is with chunk size 65536 and schedule
type static for the sample we used. In general, we would want to set the chunk size according to the parameters - some parameters may call for a tiny process and this chunk size would be too large. However, it is agreeable that static was the better scheduling type.
\\ \\
As reference, the following image shows our code's resulting Mandelbrot image.
\includepdf{Rplots.pdf}

\end{document}
